%-------------------------------------------------------------------------------
%	CAPITOLO 38
%-------------------------------------------------------------------------------

\chapter{Un uomo d'affari intraprendente}
Anche questo era un bel tipo. Alto, robusto, dritto, piedi enormi, ginocchie indentro, testa alta, cappello sulla cima della testa, barbuto come un guerriero barbaro. Aveva la mania di essere molto grande, credeva o pretendeva di avere molti appalti di opere pubbliche, corrispondenze ed affari, con anche conquiste femminili e più si scalmanava per ritenersi grande e meno era preso sul serio.\\
Appena si sentiva il treno partire per \index[Luoghi]{Ravenna}Ravenna, sul duro selciato si sentiva un correre e battere forte di tacchi, come un cavallo al trotto.\\
La gente usciva per vedere quel che succedeva... era il nostro bel tipo che ritornava dalla parte della stazione... annunziando a tutti che doveva recarsi a \index[Luoghi]{Ravenna}Ravenna, per grossi affari (che erano bugie) e che aveva perduto la corsa.\\
Tutti ridevano... e facevano finta di compassionare la sua cattiva sfortuna... per farlo parla dei suoi molti castelli in aria... ingegnosamente architettati.\\
Diceva di spendere quattro mila lire all'anno nella posta, allora che le cartoline postali valevano due soldi, i bolli per le lettere quattro, i telegrammi venti. Non sapeva molto di lettere, non aveva né buste, né carta intestata, non era tanto avanti col progresso... e si crede che per scrivere una cartolina dovesse ricorrere ai suoi figli. \\
Al mattino girava tutta la piazza per mostrare la provvista in una grande sporta... perché si sapesse come se la trattava... e per essere tenuto tra le più ricche famiglie.\\
Apriamo una parentesi per un fatto curioso. \index[Personaggi]{Faccani Rodolfo (collettore postale)}Rodolfo, collettore postale, conosceva la calligrafia di tutti, era scontroso e salace. Un bel giorno il nostro protocollista, innanzi alla posta, aveva cercato di radunare un crocchio di conoscenti e li teneva fermi con chiacchiere, col motivo nascosto di provare loro la sua grande attività affaristica, costrittiva. \\
Rodolfo\index[Personaggi]{Faccani Rodolfo (collettore postale)} esce dalla posta.\\
Il nostro .\:.\:.\: "Rodolfo, aiel gnint par me?"\footnote{"Rodolfo, c'è niente per me?"}\\
Rodolfo\index[Personaggi]{Faccani Rodolfo (collettore postale)}: "Uiè cl'la lettra c'av si scret vo ades, aspitì c'av la deg subit..."\footnote{"C'è quella lettera che vi siete scritto voi, aspettate che ve la dò subito..."}.\\
Voltò le spalle, entrò nella posta, uscì con la lettera ed al nostro .\:.\:.\: : "A vò! Lè pu la vostra..."\\
Il grande affarista rimase muto come un baccalà tra le risate e motteggi\footnote{Punzecchiare con allusioni maliziose} dei presenti.\\
In certa epoca aveva messo gli occhi di triglia, addosso ad una bella contadina bruna, certa \index[Personaggi]{Bice}Bice. Non smetteva mai le sue profferte d'amore... e di danaro ed un giorno volle che la bella gli promettesse d'andare con lui a \index[Luoghi]{Ravenna}Ravenna... per consacrarsi all'amore.\\
La bella \index[Personaggi]{Bice}Bice, promise ad un fatto, che il suo adoratore doveva ringiovanire col taglio della barba vestito da milord.\\
Così, sbarbato, ben vestito pochi giorni dopo si presenta ansioso alla sua bella, per prendere gli ultimi accordi...\\
La bella \index[Personaggi]{Bice}Bice, ridendo, rispose invece molto in disaccordo al suo filarino... precisamente così: "Ora senza barba, non vede come siete brutto... siete più salame di prima" e l'idillio finì.\\
Sempre in materia di femmine e conquiste al nostro uomo gliene capitò un'altra ancora più bella.\\
Tra le inquiline di una casa che teneva in affitto ce n'era una molto... libera. L'intraprendente nostro uomo si vantava di averla posseduta. La femmina però gliela fece pagare a suon di denaro... mandandolo scornato\footnote{Umiliato, deriso}. Non gli pagò gli otto scudi del fitto. Così il millantatore la citò innanzi al conciliatore, che era il vecchio \index[Personaggi]{Alberani Anselmo}Alberani.\\
Qui riferiamo il colloquio:\\
Conciliatore: "Tu... hai citato la .\:.\:. perché devi avere otto scudi per il fitto.\\
Intraprendente: "Se"\\
Conciliatore, alla donna: "E voi che cosa dovete dire?"\\
Donna: "Sgnor, lò, -indicando l'intraprendente- e va pu a di cun tot c'um ha .\:.\:. questa che qua un l'à da paghè"\footnote{Signore, lui, va a dire con tutti che mi ha .\:.\:. questa qui non la deve pagare!} con gesto espressivo e forte a palmo aperto sulla parte colpita...\\
L'intraprendente ammutolì.\\
Conciliatore: "Allora siete pari" e tra le risate generali dell'uditorio finì la disputa... per la quale è da dire che il nostro grande intraprendete perdè otto scudi, senza nulla avere goduto. 



