%-------------------------------------------------------------------------------
%	CAPITOLO 36
%-------------------------------------------------------------------------------

\chapter{L'atto di divisione tra cognati}
Il notaio \index[Personaggi]{Pirazzoli (notaio)}Pirazzoli si era seduto, lo scrivano \index[Personaggi]{Peppino (scrivano)}Peppino aveva steso la carta bollata sul tavolo, tre cognati e due consorti siedevano intorno al tavolo. \\
Un cognato: "Alora te t'am dé zeczent frenc..."\footnote{"Allora te mi dai cinquecento lire"}\\
Il secondo cognato: "No ad deg sol zent scud..."\footnote{"No ti dò solo cento scudi"}\\
Il primo cognato: "L'è pù l'istes."\footnote{"È lo stesso" infatti 1 scudo equivaleva a 5 lire}
Il secondo cognato: "No a ti darò me e zeczent frenc..."\footnote{"Te li dò io le cinquecento lire..."} tira fuori un lungo stile e si avventa sul primo cognato, che infila la porta e fugge a gambe levate inseguito...\\
Terzo cognato e sorelle: "Un spò scorrar cun cucalà"\footnote{"Non si può parlare con quello là"}\\
Notaio: "Me ne vado anch'io finché la strada è libera e buona..."\\
\index[Personaggi]{Peppino (scrivano)}Peppino, raccolse la carta e l'infilò nella cartella e via... dietro al notaio.\\
Un'altra volta il primo cognato, sempre per questione di somme, nelle quali era profondo aveva disteso un piazza su di una panca certi lavori.\\
Un cliente: "Ti dò una lira e mezzo"\\
Primo cognato: "No, a voi trenta bulè"\footnote{"No voglio trenta soldi" 1 soldo era equivalente a 5 centesimi di lira, quindi 30 soldi erano esattamente 1 lira e mezzo}\\
Cliente: "T'si mat..."\footnote{"Sei matto..."}\\
Primo cognato credendosi preso in giro: "A ti darò me la lira e mezzo..."\footnote{"Te la darò io la lira e mezzo..."} e gli si avventò per picchiarlo.\\
È da sperare che questo bravo primo cognato non venga mandato dai sindacati, per turno di lavoro, a prestare la sua opera alla commissione dei cambi con l'estero.\\

