%-------------------------------------------------------------------------------
%	CAPITOLO 22
%-------------------------------------------------------------------------------

\chapter{La parola - Virtù del vino}
\subsection{La parola}
Si trattava di un buon diavolo, del tipo più vecchio che nuovo, per i suoi tempi. G. \: \: sensale da vino\footnote{Mediatore in contrattazioni di prodotti enologici}, reduce delle patrie battaglie era abbastanza introdotto\footnote{Disponeva di conoscenze o relazioni utili allo svolgimento della propria attività}, quantunque non fosse troppo magniloquente\footnote{Sebbene non fosse un gran oratore}.\\
Un giorno nella cantina \index[Personaggi]{Mingazzi Fedele}Mingazzi\footnote{Fedele Mingazzi, nonno di Stefano Mingazzi, fu il veterinario del pubblico macello di Alfonsine. Investì numerose cariche nel Comune}, certo \index[Personaggi]{Allegri}Allegri di \index[Luoghi]{Glorie}Glorie doveva comprare una partita di vino. Il venditore chiedeva novanta lire, l'acquirente storceva il collo e premetteva meno, il sensale, con voce da basso profondo chiede: "La parola a me", prende le mani del venditore e dell'acquirente e con le sue le stringe fortemente imprimendo il colpo, finale solito, per la conclusione e sillaba: "Zdot scud\footnote{"Diciotto scudi". Equivalevano a 90 Lire: con l'avvento del sistema decimale nella monetazione dell'Ottocento il termine scudo venne utilizzato per la moneta da 5 lire in argento. Monete di questo modulo sono rimaste in uso fino alla prima guerra mondiale}".\\
Una risata accolse tale sproposito, ma il contratto si fece su altre basi.
\subsection{Virtù del vino}
È sempre il nostro uomo che parla.\\
Un giorno parlando con una maestra chiese: "L'an ha mai avù fiùl\footnote{"Non ha mai avuto figli?"}?"\\
La maestra: "No."\\
Il nostro G. \:\:\:\:: "Alora cla beva de ven négar gros sl'in vó mètar in sê\footnote{"Allora beva del vino nero grosso se ne vuol mettere insieme"}".\\
Farmacisti e medici aprite questa ricetta alla terapia. 
