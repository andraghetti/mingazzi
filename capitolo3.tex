%-------------------------------------------------------------------------------
%	CAPITOLO 3
%-------------------------------------------------------------------------------

\chapter{Un Stichèt}
Il detto \index[Personaggi]{Barèla}Barèla era un macellaio e venditore di carne di pecora. Aveva le sopracciglie folte che gli coprivano gli occhi neri e fondi, i baffi spioventi come due ali d'uccello disarticolate, la faccia rossa e buia, perché sempre poco illuminata dalla scarsa istruzione e mentalità. La grande manifestazione dell'animo suo semplice, che si crede sempre deluso, era la stizza\footnote{Viva irritazione, per lo più momentanea, provocata da un senso di fastidio o di molestia}. \\
Un giorno era andato a Ravenna con chi non gli capitava spesso, e così, fuori del suo ambiente e conoscenze, era impacciato, taciturno e ruminava\footnote{Pensava} molte piccole fantasie nella sua testa, senza sapere con chi sfogarsi.\\
Andò in un'osteria e sentì un tale ordinare una pietanza con un nome nuovo, strano e fuori del suo dizionario. Rimase di stucco quando vide la pietanza: una magnifica bistecca, da fargli venire l'acquolina in bocca e, confacente al suo gusto, per trincare poi un buon bicchiere di quello nero\footnote{Un bicchiere di vino rosso}, credè di evolversi dall'oblicismo\footnote{Voleva evolvere dalla sua ignoranza dicendo al cameriere "un stichèt" credendo di aver ordinato la bistecca} ed ordinò al cameriere 'un stichèt\footnote{Stuzzicadenti}'. \\
Dopo una lunga attesa ed un lungo assaporamento fantastico richiamò il camerier e gli chiese di portargli "e stichèt."\\
Il cameriere, servizievole disse "Eccolo, ve l'ho pur portato" e glielo indicò nel solito piattello sul tavolo.\\
\index[Personaggi]{Barèla}Barèla schizzò veleno, fulminò il cameriere con lo sguardo, trasformato più del solito e di balzo rispose: "Côssa vut ca megna, di bachèt?"\footnote{Cosa vuoi che mangi, degli stecchi?}
Il cameriere soggiunse: "Alora côssa vôl?"\footnote{E allora cosa vuole?}
Imbarazzo di \index[Personaggi]{Barèla}Barèla, che richiuse gli occhi, voltò le grandi sopracciglia, la bocca sotto i baffi, si sbiadì, e con voce cavernosa rispose: \\
"A vòi un quèl còm cl'a magné che sgnór che l'è."\footnote{Voglio una cosa come quella che ha mangiato quel signore lì} ed alle parole aggiunse l'indicazione fremente con pugno chiuso.\\
Il cameriere: "Allora deve dire un bistecco!"\\
\index[Personaggi]{Barèla}Barèla confuso e mortificato, forse temendo di dire un altro strafalcione grugnì: "Si" e chinò il mento sul petto per non più fiatare.
\begin{center}
\rule{1.5cm}{0.4pt}
\end{center}
Un'altra volta il buon \index[Personaggi]{Barèla}Barèla, che come avrete capito era un semplicione piovuto dal cielo, capitò in un'osteria di Ravenna mentre un suo amico e paesano (un \index[Personaggi]{Fiorentini}Fiorentini) burlone, terminava una cotoletta coi tartufi.\\
All'odore e alla bella vista del prelibato piatto, chiese all'amico quel che mangiasse e quanto costava.\\
L'amico gli rispose che costava due soldi: poi terminò, pagò, se ne andò, mentre \index[Personaggi]{Barèla}Barèla aveva mangiato la prima ed ordinava la seconda, dicendo "L'è bóna\footnote{È buona}"\\
Alla seconda, aggiunse la terza e la quarta cotoletta e tirò fuori gli otto baiocchi\footnote{Soldi} per darli al cameriere.\\
Questi\footnote{Il cameriere}, sorpreso, disse "Che cosa devo fare?"\\
\index[Personaggi]{Barèla}Barèla rispose: "Ma per le cotolette"\\
E il cameriere: "Ma costano 8 soldi l'una... quattro per otto trentadue, poi il vino... il pane... "\\
\index[Personaggi]{Barèla}Barèla frastornato: "Mo côssa dit?\footnote{Ma cosa dici?} ... an gosta du suld?"\\
Cameriere: "No, otto."\\
\index[Personaggi]{Barèla}Barèla: "Vigliac d'un amig quênt cum'à fàt spendar. A putéva magném sol una brasula!\footnote{Vigliacco d'un amico, quanto mi ha fatto spendere. Potevo mangiarmi solo una bracciola!}" 