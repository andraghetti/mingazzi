%-------------------------------------------------------------------------------
%	CAPITOLO 19
%-------------------------------------------------------------------------------

\chapter{La commedia di Ciro - In pensione}
\subsection{La commedia di Ciro}
\index[Personaggi]{Bonfiglioli Ciro}Ciro Bonfiglioli, maestro elementare, poeta, cacciatore, venditore di pistole, fucili, orologi, per 50 anni è stato una tipica macchietta\footnote{Un personaggio particolare}.\\
\indent Si vedeva un gran testone con due protuberanze sulla testa, un naso ammaccato dal quale si staccavano le narici, sotto un pelame di baffi biondi che gli coprivano una bocca sdentata, un mento sporgente, due occhi che si chiudevano a membrana come un giocattolo di Norimberga\footnote{Città tedesca famosa per i giocattoli}. Sotto un corpo lungo, lungo da persona alta e due gambine corte da fanciullo.\\
\indent In primavera vendeva i vestiti d'inverno e comprava a credito quelli d'estate. Spendeva lo stipendio, depurato dall'ammortamento dei debiti, nei primi tre giorni del mare e per gli altri 27 vivacchiava a credito presso i contadini, a riparare orologi, scrivere lettere od altro servizio. \\
\indent Aveva una grande facilità di verseggiare\footnote{Era un abile oratore}, era un gaudente\footnote{Ricercatore assiduo degli agi e dei piaceri che la vita può offrire}, rideva sempre, pigliava tutto in ischerzo, ma se ne aveva a male seriamente quando i ragazzi dicevano: \\\\
\textcal \Huge
	\centerline{Ciro, Ciro pesta pevar}
	\centerline{tu l'amzeta e pu dam da bevar}\normalfont \normalsize\footnote{"Ciro, Ciro pesta pepe, prendi la mezzetta(di vino) e poi dammi da bere"}
\index[Personaggi]{Bonfiglioli Ciro}

Ciascuno ha le sue debolezze, come poteva essere quello dello scrivere, ma però era una cosa amena\footnote{Dilettevole, divertente} per lui e per tutti. \\
\indent Verso il 1886 c'era qui nell'unico teatro, \index[Luoghi]{Teatro Camerani}Camerani\footnote{\textbf{Teatro Camerani}, che si trovava alla destra del Senio, dove oggi c'è la casa Natali subito sotto la rampa. Tale teatro fu costruito probabilmente nei primi dell'800 da \index[Personaggi]{Camerani Giovan Antonio (governatore)}Giovan Antonio Camerani, avvocato e Giudice di Pace, figlio di Matteo Camerani, fattore della famiglia Spreti che aveva sposato la sorella di Vincenzo Monti, Maria Cristina} la compagnia drammatica \index[Personaggi]{Catastini Cesare}Cesare Catastini. Il nostro Ciro\index[Personaggi]{Bonfiglioli Ciro} scrisse la commedia: "Caccia, bugie, amore". La recita venne troncata da una salva\footnote{Insieme di più colpi sparati da più bocche da fuoco} di fischi ed il telone calò.\\
\indent Il capo comico venne alla ribalta scusando la compagnia e tenendone alto il merito e scagliandosi contro quel cane dell'autore della commedia.\\
\indent Fu applaudito e scusato. Al nostro Ciro\index[Personaggi]{Bonfiglioli Ciro}, bruciava la sconfitta artistica ed agli applausi al capo comico, ardì comparire alla ribalta annunziando: <<Quei cani dei commedianti hanno rovinato la mia commedia>>.\\
\indent Una salva di fischi lo accolse. Dopo seguì una lunga invettiva tra il capo comico e l'autore, con godimento del pubblico che ne fece una carnevalata e la commedia si può dire che fu esilarante e bellissima... a telone calato. 

\subsection{Ciro in pensione}
L'ispettore scolastico \index[Personaggi]{Zaccaria Antonio}Zaccaria\footnote{Cav. prof. \textbf{Antonio Zaccaria} (1842 - 1905), ispettore scolastico per il circondario di Ravenna}, uomo alto, con barba, vestito di nero, tutto compreso del suo ufficio, col cavallo di S. Francesco\footnote{Modo di dire: `andare a piedi'} un giorno si recò al \index[Luoghi]{Fiumazzo}Fiumazzo a visitare la scuola del maestro Ciro\index[Personaggi]{Bonfiglioli Ciro}.\\
\indent Un baccano infernale si sentiva nell'aula. L'ispettore bussa, ribussa, seguita il baccano e nessuno risponde.\\
\indent Pensa l'ispettore che nell'aula non ci sia il maestro, si decide, apre la porta ed entra.\\
\indent Che spettacolo! i ragazzi giocavano, spaccavano legna, uno faceva la minestra sul tavolo del maestro, e sullo stesso tavolo era una mezzetta di vino, un bicchiere vuoto ed il maestro chino e addormentato.\\
\indent All'entrata dell'ispettore, col silenzio che fecero i ragazzi, il maestro si alzò, si stirò, arrossì, scattò in piedi, cominciò ad urlare all'ispettore: <<Ah, lei viene per rovinarmi, fuori, fuori...>> e lo spinse fuori dalla porta.\\
\indent Pochi giorni dopo Ciro\index[Personaggi]{Bonfiglioli Ciro} fu collocato in pensione d'autorità con 40 lire al mese.

\subsection{Poesie}
Quando il povero Ciro\index[Personaggi]{Bonfiglioli Ciro} era in bolletta mandava poesie ad autorità, persone facoltose, per essere aiutato.\\
\indent Una volta mandò una poesia a Sua Maestà la \index[Personaggi]{Margherita Maria Teresa Giovanna di Savoia}Regina Margherita\footnote{Margherita Maria Teresa Giovanna di Savoia (Torino, 20 novembre 1851 – Bordighera, 4 gennaio 1926) come consorte di re Umberto I} ed un'altra al Carlino. Il Carlino la mise nella cronaca burlesca e Ciro se ne ebbe a male e rispose al Carlino: <<Per una poesia la Regina mi ha mandato cinquanta lire per il delegato di P. Sicurezza\footnote{Pubblica Sicurezza, complesso di apparati, autorità e strutture preposte alla tutela dell'ordine pubblico e all'incolumità delle persone.}>>\\
\indent Subito il Carlino pubblicò: <<Sua Maestà la Regina Margherita, ha mandato il delegato di Pubblica Sicurezza dal signor Ciro Bonfiglioli\index[Personaggi]{Bonfiglioli Ciro} con ordine di arrestarlo e metterlo per sempre in prigione se ardirà di scrivere altre poesie.>>\\
\indent In altra occasione il nostro poeta andava a sciacquare i pennelli da un imbianchino, pittore ed esclamò: <<Non vedi che questi ragazzi dipinti sembrano gravidi!>>\\
\indent Il pittore fece una zirudella\footnote{Filastrocca} a Ciro per risposta e Ciro altre al pittore, e della girale solo si ricorda\footnote{Il pittore e Ciro si scambiavano filastrocche e dello scambio di `offese' si ricorda solamente quanto segue}:\\\\
\textcal \Huge
	\centerline{Quegli è Morandi, vate e pittore}
	\centerline{carnevalesco decoratore ecc...} 
\normalfont \normalsize
\\

In altra occasione, tra le tante inviò una poesia, intenta ad avere un sussidio da un agente di campagna. \\
\indent Questo, furbo, rispose: <<Caro Ciro, ho venduto la vostra poesia, ho preso lire zero, e zero ve li mando>>.\\
\indent Il nostro poeta rispose: <<C. O. agente di M... è stato un bravo topo - nei grandi magazzini - l'anghusto somarone\footnote{Somarone meschino} ecc...>>\\
\indent Così queste facezie divertivano il paese, nelle more\footnote{Solite} degli avvelenamenti politici. 

































%
