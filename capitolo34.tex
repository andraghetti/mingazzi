%-------------------------------------------------------------------------------
%	CAPITOLO 34
%-------------------------------------------------------------------------------

\chapter{Un altro originale - <<Che, che sicurezza la Gigia l'è la mi>>}
Il procaccia postale, prima che la ferrovia fosse attivata nel 1889, andava a prendere e portare tutta la corrispondenza a \index[Luoghi]{Ravenna}Ravenna giornalmente.\\
\indent Col progresso e le strade migliorate, maggior movimento anche il procaccia invece di fare il cammino a piedi, acquistò un ronzino e faceva servizio di trasporto anche per passeggeri, con un cavorino, \textcal{L}\normalfont \:\:\:2\footnote{Un cavorino, detto anche cavurrino, era la carta moneta da lire 2 che portava il ritratto del ministro Camillo Benso Conte di Cavour}, nei primi tempi, 1 scudo (\textcal{L}\normalfont \:\:\:5) negli ultimi, per andata e ritorno. \\
\indent Le tappe erano, partenza dall'osteria in \index[Luoghi]{Vincenzo Monti (piazza)}Piazza\footnote{Piazza Monti}, osteria alle \index[Luoghi]{Glorie}Glorie, Osteria a \index[Luoghi]{Mezzano}Mezzano, osteria della \index[Luoghi]{Camerlona}Camerlona, osteria e stallatico a \index[Luoghi]{Ravenna}Ravenna. Alle osterie tappe vi erano i clienti le commissioni e chi le voleva meglio raccomandare... pagava il quartino di buon vino, al pomeriggio, o l'acquavite alla mattina al procaccia.\\
\indent A \index[Luoghi]{Ravenna}Ravenna bisognava pure ingannare il tempo d'attesa e del riposo del ronzino... ed allora per il procaccia erano altre tappe straordinarie e bevute nelle osterie.\\
\indent Il povero procaccia non si poteva esimere da tante cortesie e libazioni\footnote{Offerta propiziatoria di vino}, il freddo, il caldo, gli facevano venire la necessità di mettere liquido in corpo... e se al ritorno non aveva una sbornia da cataletto.\\
\indent Il vecchio procaccia andò a riposo, ne fu fatto un altro giovane, ispecie perché lo potevano vantare come astemio.\\
\indent Il mestiere però si vede che vuole la sua parte nella manifestazione della vita e così il nostro astemio diventò un forte bevitore e tutte le sere aveva i calori di una solenne sbornia.\\
\indent Venne la ferrovia nel 1889 ed il servizio fu attivato tra la \index[Luoghi]{Vincenzo Monti (piazza)}Piazza e la \index[Luoghi]{Stazione di Alfonsine}Stazione, invece che per \index[Luoghi]{Ravenna}Ravenna, e per quante erano le corse.\\
\indent Le bevute furono maggiori per equipararle alle corse... ed il protagonista era diventato un automa, sempre eccitato, sotto i fumi dell'alcool, irritato, parlava da sé, inveiva ecc.\\
\indent Un giorno lo fermò il Delegato di P. S.\footnote{Delegato di Pubblica Sicurezza} e gli ordinò di portarlo alla \index[Luoghi]{Stazione di Alfonsine}Stazione. Fosse che quel Delegato non lo aveva mai pagato od altro, non ne volle sapere.\\
\indent Allora il Delegato lo apostrofò: "Sono il Delegato di P. S.!"\\
\indent Rispose il procaccia: "Che, che sicurezza, la Gigia\footnote{La sua cavalla} l'è la mi..."\footnote{"Che sicurezza, la Gigia è mia..."} frustò la cavalla e via di corsa, lasciando il Delegato a protestare, ma a piedi.\\
\indent Un'altra volta il nostro originale, aveva consacrato a Bacco forse più del solito, veniva col ronzino e carrettino per la \index[Luoghi]{Strada Sottofiume}Strada Sottofiume\footnote{Attuale via Mazzini}, verso il \index[Luoghi]{Borghetto}Borghetto e sbraitava contro qualche persona.\\
\indent Per una falsa tirata di redini fece montare sull'argine del fiume le due ruote destre del carrettino che si capovolse.\\
\indent Il nostro uomo rimase sotto il carrettino, imprigionato ed incolume, mentre le ruote all'aria seguitavano a roteare; seguitava ad imprecare, come se il caso non lo riguardasse.\\
\indent La Gigia, cavalla, ebbe il buon senso di fermarsi subito...\\
\indent Un'altra volta il nostro uomo doveva firmare la ricevuta dei dispacci nello scompartimento riservato all'ambulante postale, ma si vede che la mano non gli reggeva bene... il capo stazione diede la partenza, ed il treno partì col procaccia... che non volle smontare nelle stazioni intermedie, ma al capolinea di \index[Luoghi]{Ferrara}Ferrara.\\
\indent La Gigia, ed il servizio per il resto della giornata...rimasero sospesi... con delizia dei burloni. \\












































%
