%------------------------------------------------------------------------------
%	CAPITOLO 10
%------------------------------------------------------------------------------

\chapter{I grandi elettori}
\subsection{Filippino}
\index[Personaggi]{Filippino degli Angiolini}Filippino degli Angiolini aveva ereditato dal padre casa, poderi, un negozio di salumeria, unico nel paese, e molto redditizio, nonché, per sua disgrazia, una zucca vuota. A vederlo con tanto di cappello duro in testa, gli scoppettoni candidi alla Francesco Giuseppe\index[Personaggi]{Francesco Giuseppe I d'Austria}\footnote{Grandi basette collegate ai baffi come Francesco Giuseppe I d'Austria}, la pelle lucida alla coppale, subito appariva un signore spiantato\footnote{A prima vista sembrava un uomo privo di possibilità economiche}. Tratteneva, nel parlare, non sapeva dire e che cosa volesse dire "due e due fa quattro", uno 0 \footnote{Zero} non lo sapeva fare con un bicchiere. Con quella testa prese moglie, funzione facile per tutti, ma quando cominciò a dirigere la sua azienda, per la morte del padre, si mostrò la negazione degli affari. Una volta fece macellare 20 maiali, con un caldo terribile... così invece di poter conservare la carne, dovè\footnote{Dovette} buttarla nel fiume marcia frolla. Pretendeva di fare il signore, faceva buttare in un tino la finissima biancheria, invece di darla alla lavanderia, e non la estraeva ché quando era ammuffita e fradicia. Con questa condotta, frutto della sua poca testa, in breve si ridusse alla miseria squallida ed all'accattonaggio. Se la prese con Dio, che lo aveva rovinato, diceva, e con chi aveva mezzi, perché in nome della eguaglianza avrebbero dovuto rovinarsi per dovere di colleganza. La sua posizione di elettore, era stato nominato per censo durante i suoi bei tempi nonostante che fosse analfabeta, durante le elezioni gli montavano la testa e pretendeva di essere qualche cosa contro Dio e l'ordine sociale. \\
\indent Non sapeva come esplicare uno sfogo al suo malanimo, perciò si rivolgeva ad un esponente di quelli che erano contro Dio ed i signori e ruggente di sdegno ripeteva questo discorso:\\
\indent \index[Personaggi]{Filippino degli Angiolini}Filippino: <<Chi èl Rava?\footnote{<<Chi è Rava?>>}>>\\
\indent Interrogato: <<È un monarchico>>\\
\indent \index[Personaggi]{Filippino degli Angiolini}Filippino: <<No no, a na voi savè me, d'sim smasa o amasa\footnote{<<No, no, non lo voglio sapere, ditemi se accomoda (le case) o le scomoda>>}>>.\\
\indent Interrogato: <<No, le accomoda>>.\\
\indent \index[Personaggi]{Filippino degli Angiolini}Filippino: <<Alora a ni deg gnint, me a vut per quelli che smasano!\footnote{<<Allora non gli dò nulla, io voto per quelli che scomodano!>>}>>\\
\indent Ed arrabbiato si faceva dare una scheda del candidato sovversivo... ed andava a metterla nell'urna credendo di produrre l'effetto di una bomba.\\
\newpage
\subsection{Lazar}
\index[Personaggi]{Lazzaro del comacchiese}Lazzaro del comacchiese, pescivendolo al minuto, era diventato elettore col suffragio universale. Fu nel momento che si doveva votare con la scheda stampata, precisamente i monarchici portavano Rasponi, i repubblicani Mazzolani che aveva impresso nella scheda la foglia d'edera, Baldini che aveva la carriola. Il nostro uomo stette un'ora e mezzo d'orologio nella cabina elettorale e non fu capace di infilare la scheda nella busta per essere consegnata all'urna in forma segreta. Sbuffava, si contorceva, borbottava... quando il presidente del seggio, un giudice di Cagliari, intervenne a sollevarlo da tante fatiche, mettendo per lui la scheda nella busta.\\
\indent Così queste teste dovevano nominare i rappresentanti della Nazione. Povera Italia!