%-------------------------------------------------------------------------------
%	CAPITOLO 8
%-------------------------------------------------------------------------------

\chapter{Elogio Funebre - Le votazioni plebiscitarie - Rivalità d'amore}
\subsection{Elogio Funebre}
Il signor C. B. era uno stimato e ricco negoziante. Si deve anche dire che impersonava anche un po' troppo del suo ‘io' e dopo non vedeva altro, coi suoi interessi.\\
Rimasto vedovo, senza troppa commozione seppe tessere il seguente elogio funebre in lode della sua metà:
"Va, o mia Antonia, quando venisti in casa mia, mi portasti mille scudi ed ora con venticinque paoli\footnote{1 scudo = 10 paoli = 100 baiocchi} ti mando al cimitero."\\
\subsection{Le votazioni plebiscitarie}
Come partito politico tendeva al papalino, e non voleva sentire troppo parlare di liberali, che gli disturbavano il negozio. Tuttavia sapeva stare molto tra color che son sospesi\footnote{Sapeva stare con persone che non avevano rilevanti tendenze politiche} e non si sbilanciava molto, anche perché le sue idee erano in minoranza coi suoi accoliti.\\
Era solito andare nel ferrarese col biroccio per vendere molte botti d'acquavite. In occasione del plebiscito del 1860\footnote{Il 21 ottobre del 1860 si svolse il plebiscito per l'annessione del Regno delle Due Sicilie al Regno di Sardegna. Quel giorno il 79\% degli aventi diritto al voto si espressero per il Sì.} i conti Fuschini, accesi liberali, seppero che il Signor C. B. era tornato allora dal ferrarese ed ansiosi di notizie gli mandarono un biglietto concepito:\\
"Diteci come sono andate le votazioni nel ferrarese."\\
Il Signor C. non volle sbilanciarsi e rispose:\\
"Quando vado sul ferrarese mi occupo di vuotare le mie botti, altro non so di votazioni."\\
La spiritosa risposta provocò risate e rimase celebre. \\
\subsection{Rivalità d'amore}
Per amore e profondità nel dolce parlare Italiano il nostro Signor C. è anche rimasto celebre. Si era innamorato di una bella ragazza, lui attempato e con due baffoni spioventi come due nere saracche\footnote{La salacca, saraca o sarachina è la definizione commerciale di alcuni tipi di pesce}. \\
Il male venne, che la sua bella aveva un'altro adoratore in un giovincello sbarbato, uno sbarbatello, come si direbbe.\\
Geloso, irato contro lo sbarbatello il nostro Signor C. innanzi allo specchio, pavoneggiava la sua carattestica di fiero uomo, contro il disprezzato sbarbatello e provava le mosse da fargli per avvilirlo e le parole da dirgli; avvicinandosi allo specchio con enfasi:\\
Si tirava su i mustacchi\footnote{Baffi} ricciandoli, poi sghignazzante: "... e di questi voi non ne abbiate."\\
Secondo il suo ‘io' lui era un uomo il rivale invece un fanciullo a balia.


