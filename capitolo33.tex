%-------------------------------------------------------------------------------
%	CAPITOLO 33
%-------------------------------------------------------------------------------

\chapter{Un tipo originale} 
\index[Personaggi]{M. Giovanni, Giovannone}G. M.\footnote{\textbf{Giovannone M.}, il soprannome è stato definito in seguito.} detto S. d' M. era un esperto nel condurre il suo biroccio, nel governo del suo attacco e soprattutto nel vendere il suo vino. \\
\indent Vestiva con un cappello piccolo tondo, alla romagnola, alla cintola aveva la caratteristica fascia rossa dei vetturali\footnote{Incaricato di eseguire un trasporto di merci con un carro, un barroccio o una bestia da soma}, che lo teneva stretto e gli faceva saltar fuori ancor più la pancia.\\
\indent Di mezza età, si era ingrassato, come tutti i corpi sani, guance sporgenti dalle pareti del cranio, un bel parpagliolo rotondo, gli davano una certa rassomiglianza ad una pentola di terracotta.\\
\indent La sua voce da basso profondo era cavernosa, stridula, senza modulazione, tra il toro od il leone infuriato, non rideva mai ed era inclino a prendere tutto sul serio... e parla senza preamboli.\\
\indent Un giorno voleva salire in carrettino, ma si strinse una delle parti genitali, non so come. Qui si era nel tempo del boicott\index[Personaggi]{Faccani Rodolfo (barbiere)}aggio del servizio sanitario, così che il nostro protagonista, consigliato, andò a farsi visitare a \index[Luoghi]{Ravenna}Ravenna dal valente chirurgo prof \index[Personaggi]{Giannetto (professore, chirurgo)}Giannetto e testuamente riferiamo:\\
\indent \index[Personaggi]{C\:.\:.\:.}C\:.\:.\:. (moglie di G.): <<Sgnor professor al se fat mal a un testicul\footnote{<<Signor professore, si è fatto male ad un testicolo>>}>>.\\
\indent G\:.\:.\:.\index[Personaggi]{M. Giovanni, Giovannone} :<<Ignurata... mo che testicolo e mo testicolo, tan si bona dir un `marò'?\footnote{"Ignorante... ma che testicolo e testicolo, non sei capace di dire un marone?"}\\
\indent Il professore rise ed esaminò il paziente e si pronunziò: <<Qui bisogna asportare il testicolo malato.>>\\
\indent \index[Personaggi]{M. Giovanni, Giovannone}G\:.\:.\:. : <<Maiun, mo el pu bon?\footnote{<<Minchioni! Ma è però buono?>>}>>\\
\indent \index[Personaggi]{M. Giovanni, Giovannone}G\:.\:.\:. <<Mo si so me sol cun i marò stra a la casa>>\footnote{<<Ma ci sono io coi coglioni tra la cassa>>}\\
\indent \index[Personaggi]{Giannetto (professore, chirurgo)}Giannetto: <<Se non ha stima di me, vada da un altro>>\\
\indent \index[Personaggi]{M. Giovanni, Giovannone}G\:.\:.\:. : <<Quaiù mo sum castra, com'armestia amaséi>>\footnote{<<Coglioni! Ma se mi castra, come rimango accomodato?>>}\\
\vspace{0.5cm}
\centerline{\rule{1.5cm}{0.4pt}}
\vspace{0.5cm}
Una povera donna un giorno andò dal nostro \index[Personaggi]{M. Giovanni, Giovannone}G\:.\:.\:. che stava terminando il pranzo e gli chiese: <<Allo miga una camera d'afitar?>>\\
\indent \index[Personaggi]{M. Giovanni, Giovannone}G\:.\:.\:. alla moglie \index[Personaggi]{C\:.\:.\:.}C\:.\:.\:. : <<Aièla, faglia avdè.>>\footnote{<<C'è, fagliela vedere.>>}\\
\indent Vista la camera, la povera donna tornò per il responso.\\
\indent Donna: <<Una teraza, una ringhiera...>> (ma non finì)\\
\indent \index[Personaggi]{M. Giovanni, Giovannone}G\:.\:.\:. <<Andì là a si bela>>\footnote{<<Andate là, siete bella>>} scrollò le spalle e la mandò via.\\
\vspace{0.5cm}
\centerline{\rule{1.5cm}{0.4pt}}
\vspace{0.5cm}
Aveva un credito con un'oste per una fornitura di vino e non essendo pagato si fece dare un organetto.\\
\indent Successivamente vendette a credito l'organo, a persona che tardava a pagarglielo e trovato in piazza il suo nuovo debitore lo apostrofò:\\
\indent G\:.\:.\:. : <<Ui da cla ca longa, sonal sonal cl'organ?>>\footnote{<<Eih da quella casa lunga (debitore lungo) suona, suona quell'organo?>> - traduzione di Mingazzi riportata fedelmente, probabilmente "debitore lungo" era legato al fatto che questa persona tardasse a pagare il debito.}\\
\vspace{0.5cm}
\centerline{\rule{1.5cm}{0.4pt}}
\vspace{0.5cm}
Era un uomo pratico e di buon senso.\\
\indent Un giorno un tale gli disse: <<Voi siete un Signore, perché non andate ad abitare a \index[Luoghi]{Bologna}Bologna e passarvela?>>\\
\indent \index[Personaggi]{M. Giovanni, Giovannone}G\:.\:.\:. : <<A fé chè? Sa steg agl'Infulsen i dis \emph{`S... l'ha di quatrèn'} sa veg a Bulogna i dis \emph{`Chi èl cl'ignurent'}>>\footnote{<<A far che? Se sto ad \index[Luoghi]{Alfonsine}Alfonsine dicono \emph{`Giovannone ha dei quattrini'}, se vado a \index[Luoghi]{Bologna}Bologna dicono \emph{`Chi è quell'ignorante?'}>>}\\
\\
\centerline{\rule{1.5cm}{0.4pt}}\\
\\
Negli ultimi anni aveva dovuto prendere uno scrivano contabile, un simpaticissimo tipo ameno. \\
\indent Un giorno \index[Personaggi]{M. Giovanni, Giovannone}G\:.\:.\:. torna dal mercato e dice: <<Ho comprato un paio di bestie>>\\
\indent Ministro\footnote{Lo scrivano viene chiamato "Ministro" da Mingazzi}: "Dove le avete messe?"\\
\indent \index[Personaggi]{M. Giovanni, Giovannone}G\:.\:.\:. <<Cos'avliv savé vo dgl'intares d'ietar, a mi cuntiv vo i vostar?\footnote{<<Che cosa volete sentire voi gli interessi degli altri, me li raccontate i vostri?>>}>>\\
\indent Un anno dopo i nostri due facevano i conti con un contadino e nella stalla risultava un utile esagerato.\\
\indent Ministro: <<Quelle bestie che compraste quella volta dove le avete messe?>>\\
\indent \index[Personaggi]{M. Giovanni, Giovannone}G\:.\:.\:. : <<A glia avudi stu ca que\footnote{<<Le ha avute costui>>}>> indicando il colono.\\
\indent Ministro: <<È trovato l'errore.>>\\
\indent \index[Personaggi]{M. Giovanni, Giovannone}G\:.\:.\:. <<Vo a scrivì sempar, me a na so cosa ca scriviva s'en avì sgné al besti>>.\footnote{<<Voi scrivete sempre, io non so che cosa scriviate se non avete segnato le bestie>>}\\
Ministro: <<Mo s'an ma dsi, cosa avliv ca seva me>>.\footnote{<<Ma se non me lo dite, che cosa volete che sappia?>>}\\
\indent \index[Personaggi]{M. Giovanni, Giovannone}G\:.\:.\:. <<An la vi da savé vô?\footnote{<<Non lo dovete sapere voi?>>}>>\\
\\
\centerline{\rule{1.5cm}{0.4pt}}\\
\\
Un'altra volta chiamò il ministro e gli disse: <<Scrivì una cartulena a \index[Personaggi]{Pirin Bec}Pirin Bec, a e \index[Luoghi]{Ponte Albergone}Pont Albargon\footnote{<<Scrivete una cartolina a Pirin Bec, a Ponte Albergone>> - Ponte Albergone è la zona e/o il ponte che attraversa il fiume Lamone, con la \index[Luoghi]{Vecchia Albergone (via)}via Vecchia Albergone, sotto \index[Luoghi]{Traversara}Traversara.}>>\\
\indent Il Ministro, inforcata la penna e pronto: <<Come si chiama?>>\\
\indent \index[Personaggi]{M. Giovanni, Giovannone}G\:.\:.\:. : <<Vo scrivì cum cav dec: `a Pirin Bec', in cgnos tot\footnote{<<Voi scrivete come vi dico: `a Pirin Bec', lo conoscono tutti>>}>>\\
\indent Ministro: <<Volete scrivere così, se ne avrà male.>>\\
\indent \index[Personaggi]{M. Giovanni, Giovannone}G\:.\:.\:. : <<Dasim met, scrivì Pirin Bec\footnote{<<Datemi ascolto, scrivete Pirin Bec>>}>>\\
\indent E così fu fatto. E poi \index[Personaggi]{M. Giovanni, Giovannone}G\:.\:.\:. : <<Dsii s'uiè piasù e ven, slin vo dletar. Che zuba se non vien giù il giavolo aveg a la da lò\footnote{<<Chiedetegli se gli è piaciuto il vino e se ne vuole dell'altro. Questo giovedì, se non viene giù il diavolo vado là da lui.>> - `vien giù il diavolo' si dice solitamente riferendosi alle condizioni atmosferiche}>>\\
\indent Ministro: <<Un gni sta più in tla cartulena\footnote{<<Non ci sta più nella cartolina>>}>>\\
\indent \index[Personaggi]{M. Giovanni, Giovannone}G\:.\:.\:. : <<Av si fat da là a mez!\footnote{<<Avete iniziato da lì in mezzo!>> - il ministro aveva cominciato a scrivere da metà cartolina, finendo lo spazio}>>

