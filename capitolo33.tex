%-------------------------------------------------------------------------------
%	CAPITOLO 33
%-------------------------------------------------------------------------------

\chapter{Un tipo originale} 
\index[Personaggi]{M. Giovanni, Giovannone}G. M.\footnote{Giovannone M., il soprannome è stato definito più sotto} detto S. d' M. era un esperto nel condurre il suo biroccio, nel governo del suo attacco e soprattutto nel vendere il suo vino. \\
Vestiva con un cappello piccolo tondo, alla romagnola, alla cintola aveva la caratteristica fascia rossa dei vetturali\footnote{Incaricato di eseguire un trasporto di merci con un carro, un barroccio o una bestia da soma}, che lo teneva stretto e gli faceva saltar fuori ancor più la pancia.\\
Di mezza età, si era ingrassato, come tutti i corpi sani, guance sporgenti dalle pareti del cranio, un bel parpagliolo rotondo, gli davano una certa rassomiglianza alla lenia\footnote{AGGIUNGERE DEFINIZIONE, cos'è?} di pentola di terracotta.\\
La sua voce da basso profondo era cavernosa, stridula, senza modulazione, tra il toro od il leone infuriato, non rideva mai ed era inclino a prendere tutto sul serio... e parla senza preamboli.\\
Un giorno voleva salire in carrettino, ma si strinse una delle parti genitali, non so come. Qui si era nel tempo del boicott\index[Personaggi]{Faccani Rodolfo (commerciante)}aggio del servizio sanitario, così che il nostro protagonista, consigliato, andò a farsi visitare a \index[Luoghi]{Ravenna}Ravenna dal valente chirurgo prof \index[Personaggi]{Giannetto (professore, chirurgo)}Giannetto e testuamente riferiamo:
\index[Personaggi]{C\:.\:.\:.}C\:.\:.\:. (moglie di G.): "Sgnor professor al se fat mal a un testicul\footnote{"Signor professore, si è fatto male ad un testicolo"}.\\
G\:.\:.\:.\index[Personaggi]{M. Giovanni, Giovannone} :"ignurata... mo che testicolo e no testicolo, tan si bona dir un m...\footnote{"Ignorante... ma che testicolo e non testicolo, non sei capace di dire un marone?"}\\
Il professore rise ed esaminò il paziente e si pronunziò: "Qui bisogna asportare il testicolo malato..."\\
\index[Personaggi]{M. Giovanni, Giovannone}G\:.\:.\:. : "Maiun, mo el pu bon?\footnote{"Minchioni, ma è però buono?"}"
\index[Personaggi]{M. Giovanni, Giovannone}G\:.\:.\:. "Mo si so me sol cun i m... stra a la casa"\footnote{"Ma ci sono io coi coglioni tra la cassa"}\\
\index[Personaggi]{Giannetto (professore, chirurgo)}Giannetto: "Se non ha stima di me, vada da un altro"\\
\index[Personaggi]{M. Giovanni, Giovannone}G\:.\:.\:. : "Quaiù mo sum castra, com'armestia amaséi"\footnote{"Coglione ma se mi castra, come rimango accomodato"}\\
\\
\centerline{\rule{1.5cm}{0.4pt}}
\\\\
Una povera donna un giorno andò dal nostro \index[Personaggi]{M. Giovanni, Giovannone}G\:.\:.\:. che stava terminando il pranzo e gli chiese: "Allo miga una camera d'afitar?"\\
\index[Personaggi]{M. Giovanni, Giovannone}G\:.\:.\:. alla moglie \index[Personaggi]{C\:.\:.\:.}C\:.\:.\:. : "Aièla, faglia avdè."\footnote{"C'è, fagliela vedere."}\\
Vista la camera la povera donna tornò per il responso.\\
Donna: "Una teraza, una ringhiera..." (ma non finì)\\
\index[Personaggi]{M. Giovanni, Giovannone}G\:.\:.\:. "Andì là a si bela"\footnote{"Andate là siete bella"} scrollò le spalle e la mandò via. \footnote{Non ho mica capito niente io}\\
\\
\centerline{\rule{1.5cm}{0.4pt}}\\
\\
Aveva un credito con un'oste per una fornitura di vino e non essendo pagato si fece dare un organetto.\\
Successivamente vendette a credito l'organo, a persona che tardava a pagarglielo e trovato in piazza il suo nuovo debitore lo apostrofò:\\
G\:.\:.\:. : "Ui da cla ca longa, sonal sonal cl'organ?"\footnote{Mingazzi ha scritto "Eih da quella casa lunga (debitore lungo) suona, suona quell'organo?" ma non capisco cosa intenda}\\
\\
\centerline{\rule{1.5cm}{0.4pt}}\\
\\
Era un uomo pratico e di buon senso. Un giorno un tale gli disse: "Voi siete un Signore, perché non andate ad abitare a \index[Luoghi]{Bologna}Bologna e passarvela?"\\
\index[Personaggi]{M. Giovanni, Giovannone}G\:.\:.\:. : "A fé chè? Sa steg agl'Infulsen i dis \emph{‘S... l'ha di quatrèn'} sa veg a Bulogna i dis \emph{‘Chi èl cl'ignurent'} "\footnote{"A far che? Se sto ad \index[Luoghi]{Alfonsine}Alfonsine dicono \emph{'Giovannone ha dei quattrini'}, se vado a \index[Luoghi]{Bologna}Bologna dicono \emph{‘chi è quell'ignorante?'} "}\\
\\
\centerline{\rule{1.5cm}{0.4pt}}\\
\\
Negli ultimi anni aveva dovuto prendere uno scrivano contabile, un simpaticissimo tipo ameno. \\
Un giorno \index[Personaggi]{M. Giovanni, Giovannone}G\:.\:.\:. torna dal mercato e dice: "Ho comprato un paio di bestie"\\
Ministro\footnote{Lo scrivano viene chiamato "Ministro" da Mingazzi}: "Dove le avete messe?"\\
\index[Personaggi]{M. Giovanni, Giovannone}G\:.\:.\:. "Cos'avliv savé vo intares d'ietar, a mi cuntiv vo i vostar?"\footnote{"Che cosa volete sentire voi gli interessi degli altri, me li raccontate i vostri?"}\\
Un anno dopo i nostri due facevano i conti con un contadino e nella stalla risultava un utile esagerato.\\
Ministro: "Quelle bestie che compraste quella volta dove le avete messe?"\\
\index[Personaggi]{M. Giovanni, Giovannone}G\:.\:.\:. : "A glia avudi stu ca que"\footnote{"Le ha avute costui"} indicando il colono.\\
Ministro: "È trovato l'errore."\\
\index[Personaggi]{M. Giovanni, Giovannone}G\:.\:.\:. "Vo a scrivì sempar, me a na so cosa ca scriviva s'en avì sgné al besti".\footnote{"Voi scrivete sempre, io non so che cosa scriviate se non avete segnato le bestie"}\\
Ministro: "Mo s'an ma dsi cosa avliv ca seva me".\footnote{"Ma se non me lo dite che cosa volete che sappia"}\\
\index[Personaggi]{M. Giovanni, Giovannone}G\:.\:.\:. "An la vi da savé vô?"\footnote{"Non lo dovete sapere voi?"}\\
\\
\centerline{\rule{1.5cm}{0.4pt}}\\
\\
Un'altra volta chiamò il ministro e gli disse: "Scrivì una cartulena a \index[Personaggi]{Pirin Bec}Pirin Bec, a e \index[Luoghi]{Ponte Albergone}Pont Albargon a Cuper"\footnote{"Scrivete una cartolina a Pirin Bec, a Ponte Albergone SISTEMARE}\\
Il Ministro, inforcata la penna e pronto: "come si chiama?"\\
\index[Personaggi]{M. Giovanni, Giovannone}G\:.\:.\:. : "Vo scrivì cum cav dec: 'a Pirin Bec', in cgnos tot"\footnote{"Voi scrivete come vi dico: 'a Pirin Bec', lo conoscono tutti"}\\
Ministro: "Volete scrivere così, se ne avrà male."\\
\index[Personaggi]{M. Giovanni, Giovannone}G\:.\:.\:. : "Dasim met, scrivì Pirin Bec"\footnote{"Datemi ascolto, scrivete Pirin Bec"}\\
E così fu fatto. E poi \index[Personaggi]{M. Giovanni, Giovannone}G\:.\:.\:. : "Dsii s'uiè piasù e ven, slin vo dletar. Che zuba se non vien giù il giavolo aveg a la da lò"\footnote{"Chiedetegli se gli è piaciuto il vino e se ne vuole dell'altro. Questo giovedì, se non viene giù il diavolo vado là da lui." ('vien giù il diavolo' si dice solitamente riferendosi alle condizioni atmosferiche) SISTEMARE: era ovvio anche per chi non è romagnolo?}\\
Ministro: "un gni sta più in tla cartulena"\footnote{"Non ci sta più nella cartolina"}\\
\index[Personaggi]{M. Giovanni, Giovannone}G\:.\:.\:. : "Av si fat da là a mez!"\footnote{"Avete iniziato da lì in mezzo!" il ministro aveva cominciato a scrivere da metà cartolina, finendo lo spazio}

