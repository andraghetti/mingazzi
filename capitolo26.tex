%-------------------------------------------------------------------------------
%	CAPITOLO 26
%-------------------------------------------------------------------------------

\chapter{I piselli e la ruvaglia}
M\:.\:.\:.\footnote{Romano Gagliardi, mio bisnonno, che lesse questi manoscritti dopo la morte di suo cugino, Stefano Mingazzi, scrisse un appunto sopra la `M': "\textbf{Mauro Ghetti}". Risulta inoltre che nel 1912 un certo Mauro Ghetti possedesse una locanda ad Alfonsine, quindi ho sostituito `Mauro' alla M che aveva scritto Mingazzi.}\index[Personaggi]{Ghetti Mauro} era padrone dell'osteria e stallatico.\\
\indent A quell'osteria si mangiava divinamente bene, l'umido poi faceva resuscitare i morti... ed è rimasto famoso. Era tutto merito delle donne della famiglia, specialmente della vecchia.\\
\indent Il nostro Mauro\index[Personaggi]{Ghetti Mauro} badava allo stallatico, acquisto del vino ecc. e l'ambiente di casa poco si confaceva alla sua abitudine di saper ben governare i cavalli e di signoreggiare nella stalla. Era venuta la ferrovia e si cominciava a vedere qualche forestiero, che non parlava che l'Italiano.\\
\indent Un giorno capita uno di questi tipi di forestieri e le donne di cucina, impedite mandano il buon Mauro\index[Personaggi]{Ghetti Mauro}\index[Personaggi]{Ghetti Mauro} per servirlo.\\
\indent Facciamo subito la presentazione.\\
\indent Mauro\index[Personaggi]{Ghetti Mauro} si presenta in manica di camicia, grande fascia rossa legata alla cinta, sulla quale saltava fuori una gran pancia e più che pancia, stomaco. Cappello in testa... doveva parlare... cercava la parola, ed intanto si grattava la testa con le unghie, si calcava e sollevava il cappello di testa... finalmente. \\
\indent Mauro\index[Personaggi]{Ghetti Mauro}: <<Cosa vol?\footnote{<<Cosa vuole?>>}>>\\
\indent Forestiero: <<Una minestra coi piselli>>.\\
\indent Mauro:\index[Personaggi]{Ghetti Mauro} <<An n'avèn brisa\footnote{<<Non ne abbiamo>>}>>.\\
\indent Forestiero: <<Allora mi porti ecc>>.\\
\indent Mauro\index[Personaggi]{Ghetti Mauro} volta le spalle va in cucina a dare gli ordini alle donne. \\
\indent <<Cucalà d'che sfrinziè l'avleva d'amnestra con i piselli. Us ved che magna sol di curadèn d'zinzéla... e me ai ò dèt ch'en na ven\footnote{<<Quel raffinatello là voleva della minestra coi piselli. - Si vede che mangia solo delle coratelle di zanzara... ed io gli ho detto che non ne abbiamo>>. Si suol dire "mangia solo corratelle di zanzara" a chi è un pò raffinato di gusto}. \\
\indent Donne: <<Ma cosa, ne abbiamo pure! L'è l'arveaia\footnote{<<Ma cosa, ne abbiamo pure! È la ruvaglia!>> La ruvaglia è un tipo di legume chiamato anche roveja simile al pisello e dal seme colorato che va dal verde scuro al marrone grigio. Nel secoli scorsi era coltivato e consumato in abbondanza, ma col tempo se n'è perso l'utilizzo fino alla quasi estinzione. Questa è una testimonianza dell'utilizzo della ruvaglia in Romagna.}>>.\\
\indent Mauro sorpreso ed adirato: <<Mo cl'ignurant un'era bon `d di cun la ruvaglia?!\footnote{<<Ma quell'ignorante non era capace di dire `con la ruvaglia'>>}>>\\
\indent Morale: in ogni paese che vai, và col dizionario in tasca... adatto secondo le teste.

