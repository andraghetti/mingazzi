%-------------------------------------------------------------------------------
%	CAPITOLO 37
%-------------------------------------------------------------------------------

\chapter{Un telegramma}
Il buon \index[Personaggi]{M. P.}P. M. aveva un figlio all'università, con tutte le preoccupazioni della spesa da sostenere e delle bagatelle che esso figlio gli poteva procurare per la sua gioventù e vivacità.\\
Un giorno il signor \index[Personaggi]{M. P.}P.\:.\:.\: si vide recapitare un telegramma, vergato a grossi caratteri dal collettore locale postegrafico \index[Personaggi]{Mercatelli (collettore postale)}Barandellaccio di Mercatelli, così:\\ \textcal \Huge\\
\centerline {Vostro figlio carcerato}\\
\centerline {firmato da persona amica}\\ \normalfont \normalsize \\
Il povero uomo, cominciò a disperarsi, in fretta e furia riempire il portafoglio, perché in qualunque disgrazia bisogna cominciare a vuotarlo... attacca il cavallo e via a \index[Luoghi]{Lugo}Lugo per prendere il treno per \index[Luoghi]{Bologna}Bologna.\\
Il lettore potrà bene immaginare i tristi pensieri che accompagnavano il nostro uomo fino a \index[Luoghi]{Bologna}Bologna... dove arrivò più morto che vivo.\\
Appena arrivato a Bologna, trafilato si recò all'abitazione del figlio e trovata la padrona di casa subito l'apostrofò: "Mio figlio?"\\
Padrona: "È fuori in baldoria con gli amici."\\
\index[Personaggi]{M. P.}P.\:.\:.\: : "Oh " Mio signore ma che cos'ha fatto che l'hanno arrestato?"\\
Padrona: "Arrestato! Non so nulla. Andiamo a vedere in trattoria."\\
In trattoria trovarono un'orgia infernale a banchetto di studenti e non c'era caso né di parlare, né di farsi ascoltare... con un pò di pazienza trovarono il sospirato figlio... laureato... e la scena cambiò... con una smunta allegra e simpatica al portafoglio di papà.\\
Quel \index[Personaggi]{Mercatelli (collettore postale)}Barandellaccio del telegrafo ne combinava delle curiose!

