%-------------------------------------------------------------------------------
%	CAPITOLO 17
%-------------------------------------------------------------------------------

\chapter{Tre teste da capestro tutte e tre}
Il vecchio conte \index[Personaggi]{Foschini Camillo (conte)}Foschini era padrone ed abitava la casa \index[Luoghi]{Alberani (palazzo)}Alberani\footnote{La casa della famiglia del Dott. \index[Personaggi]{Alberani Anselmo}Anselmo Alberani, uno dei più ricchi proprietari terrieri di Alfonsine, in via Reale (dove oggi c’è la fabbrica di trasformazione “Contarini”). Era il padre di Alberto Alberani, che fu sindaco di Alfonsine}. Una sera faceva la partita nella camera da pranzo, con amici, tra i quali \index[Personaggi]{Don Salvatori Ruggero}Don Ruggero. Era irrequieto, si contorceva sulla sedia, sembrava sugli spini. Accese un lume ad olio, disse agli amici: "vengo subito" e sparì.\\
È da dire che il vecchio conte era un accanito papalino, mentre suo figlio \index[Personaggi]{Foschini Stefano}Stefano era liberale carbonaro\footnote{La Carboneria è stata una società segreta rivoluzionaria italiana, nata nell'allora Regno di Napoli durante i primi anni dell’Ottocento su valori patriottici e liberali.} acceso. \\
In una camera superiore della casa erano a confabulare \index[Personaggi]{Farini Luigi Carlo}Luigi Carlo Farini\footnote{Luigi Carlo Farini (Russi, 22 ottobre 1812 – Quarto, 1º agosto 1866) è stato un medico, storico e politico italiano, per breve tempo Presidente del Consiglio dei ministri del Regno d'Italia tra il 1862 e il 1863.} il futuro dittatore e Ministro, \index[Personaggi]{Strocchi Girolamo}Momo Strocchi\footnote{Figlio di \index[Personaggi]{Strocchi Dionigi}Dionigi Strocchi che era stato un letterato, grecista e latinista italiano.} liberale faentino ed il conte \index[Personaggi]{Foschini Stefano}Stefano\footnote{Farini, Strocchi e Foschini erano tre rivoluzionari liberali che parteciparono al moto antipapalino di Romagna del 1843}. Il vecchio conte padre si mise ad origliare dietro la porta i colloqui. Udito che ebbe che parlavano di politica, aprì la porta come un fulmine, col lume nella sinistra, la destra e l'indice teso si rivolse ai presenti indicandoli: "Uno, due, tre, tre teste da capestro tutte tre\footnote{"Tre teste da legare/impiccare"}". Voltò le spalle, richiuse la porta e discese in fretta dagli amici, ridendo e dicendo:\\
"Credevo che facessero firmare cambiali a mio figlio ... invece parlano di politica!"\\
Dopo il temporale era venuto il sereno nella faccia del vecchio conte, le cambiali erano uno sturbo\footnote{Scompiglio} forte... trasgredire ai suoi principi politici era cosa sopportabile!

