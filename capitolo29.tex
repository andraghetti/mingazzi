%-------------------------------------------------------------------------------
%	CAPITOLO 29
%-------------------------------------------------------------------------------
%
\chapter{I maestri di lingua - Araldo il Capolega B...}
\index[Personaggi]{Lanconelli Araldo}Araldo\footnote{\textbf{Araldo Lanconelli}, (ndr) tutto ciò che so di lui è che sua figlia si chiamava Angelina e che questa fu presente al matrimonio di \index[Personaggi]{Marini Marino}Marino Marini. Tuttavia, Mingazzi ci suggerisce che Araldo coprì la carica di assessore.}, seduto, calvo, con la coppole sulla cervice, ben nutrito, baffi, spioventi, alla moda, era un bell'uomo. Dritto non ci stava perché era zoppo... e così nelle concioni\footnote{Discorsi} dei repubblicani, si vestiva di nero e si metteva in evidenza nel palco accanto a \index[Personaggi]{Mirabelli Roberto (onorevole)}Mirabelli\footnote{\textbf{Roberto Mirabelli}, onorevole, di origine calabrese, più volte deputato di Ravenna per il PRI} o \index[Personaggi]{Mazzolani Ulderico (onorevole)}Mazzolani\footnote{\textbf{Ulderico Mazzolani}, onorevole repubblicano}, essendo un bel pezzo... decorativo.\\
\indent Con le sue idee repubblicane non transigeva, era attaccato alla carica e fu anche assessore.\\
\indent Una sera, durante l'assessorato, si recò a casa, come d'abitudine all'ora di cena, del \index[Personaggi]{Meruzzi dott. Cassiano}Dr. Meruzzi\footnote{Dr. \textbf{Cassiano Meruzzi}, medico condotto di Alfonsine, cugino di Cassiano Bagnara, figlio di Giovanni, proprietario della casa di Vincenzo Monti.}. Poveretto non stava più nella pelle, rideva, si contorceva, faceva delle smorfie.\\
\indent Allora cominciò questo dialogo:\\
\indent Dr. Meruzzi\index[Personaggi]{Meruzzi dott. Cassiano}: <<Insomma, sei troppo contento... hai qualche cosa, dillo.>>\\
\indent \index[Personaggi]{Lanconelli Araldo}Araldo: <<Ecco. Noi repubblicani, a fasèn un oratorio a què drì a la mura d'Terulin\footnote{<<Noi repubblicani facciamo un oratorio qua dietro le mura di Terulin>> - (ndr) non ho trovato corrispondenze con `Berulin'}>>.\\
\indent Donne di casa: <<Oh! Bene così andiamo a pregare... è vicino.>>\\
\indent \index[Personaggi]{Lanconelli Araldo}Araldo si rabbuia e tace.\\ 
\indent \index[Personaggi]{Meruzzi dott. Cassiano}Dr. Meruzzi: <<Voi repubblicani, mangia preti, non ci sarebbe altro... che faceste proprio una chiesa.>>
\indent \index[Personaggi]{Lanconelli Araldo}Araldo in fretta: <<Ma che cisa, a fasèn un pisadur!\footnote{<<Ma che chiesa, facciamo un pisciatoio!>>}>>.\\
\indent \index[Personaggi]{Meruzzi dott. Cassiano}Meruzzi e donne risposero con una risata e poi <<Puh!>>\\
\indent \index[Personaggi]{Meruzzi dott. Cassiano}Meruzzi: <<Alora té da dì un orinatoio...\footnote{<<Allora devi dire orinatoio>>}>>\\
\indent \index[Personaggi]{Lanconelli Araldo}Araldo: <<L'è l'istes, l'è question d'paròl\footnote{<<È lo stesso è questione di parole>>}>>\\
\indent Era anche assessore alla pubblica istruzione certamente se ne intendeva molto e poteva dire anche altro.

Un altro giorno il nostro \index[Personaggi]{Lanconelli Araldo}Araldo, raccontava: <<L'ha scrèt a cà \index[Personaggi]{Forlivesi Sebastiano `Nisò d'Furlivési' (commerciante)}Nisò d'Furlivési\footnote{\textbf{Forlivesi Sebastiano}, nato nel 1830 e morto nel 1905, è il primo dei quattro alfonsinesi che in un modo o nell'altro furono coinvolti nelle varie imprese garibaldine. Partecipò fin dall'inizio alla difesa della Repubblica Romana (1849). Con l'unità d'Italia ebbe a ricompensa della sua attività garibaldina il permesso di aprire una bottega. Quella bottega, passando di figlio in figlio, è ancora oggi attiva in Corso Garibaldi ed è sempre stata chiamata "Butega dla Formazala".} cl'à sintì \index[Personaggi]{De Maria Ugo (professore)}Ugo d'De Maria\footnote{\textbf{Ugo De Maria}, libero docente dell'Università di Parlemo. Fu allievo del Carducci.} in piaza a Palermo, che siringava la folla, contra a Nasi\footnote{<<Ha scritto a casa Nisò d'Forlivesi che ha sentito Ugo De Maria in piazza a Palermo che siringava (per arringava) la folla>>}>> (era il tempo dello scandalo Nasi\footnote{Il Ministro della Pubblica Istruzione, Nunzio Nasi venne accusato di gravi irregolarità nel suo operato. Si parla di corruzione, sussidi ingiustificati, firme sospette, favoritismi, spese personali pagate con i soldi pubblici.})\\
\indent Speriamo che il buon Dio tenga lontano ai palermitani il mal d'urina! Ce ne sarebbero troppe, ma finiamo con l'ultima.\\

Un giorno, al nostro Araldo viene consegnata la scheda per il censimento della popolazione. Da letterato si mette subito all'opera.\\
\vspace{0.5cm}

\textcal \Huge
Araldo L\:.\:.\:.\: fu \:.\:.\:.\:\\
\normalfont \normalsize
di professione:
\textcal \Huge Agente Urale
\normalfont \normalsize (per `rurale')\\
di religione: 
\textcal \Huge Ateo
\normalfont \normalsize\\
\vspace{0.5cm}
\indent Speriamo bene che il nostro protagonista non vada alla storia... e che la scheda di suo pugno non vada in bella mostra... in vetrina.\\

\vspace{0.5cm}
\centerline{\rule{1.5cm}{0.4pt}}
\vspace{0.5cm}


Nel momento rosso i socialisti avevano caricato un buon uomo di molte frasi, capite come poteva e secondo lui solo\footnote{Interpretate a modo suo}.\\
\indent Necessitava una concione\footnote{Discorso solenne in pubblico}.\\
\indent Il nostro uomo, rosso, forte, gesticolante cominciava.\\
\indent <<Soci, amici, compagni, sucilèsta, d'la sucietè, d'la fratelênza, d'la adunênza, d'la cumbrècula, d'la sozia di cuntadèn.\footnote{<<Soci, amici, compagni, socialisti, della società, della fratellanza, dell'adunanza, della combriccola, della socia dei contadini>>}>> Punto e basta il repertorio era esaurito con un <<Evviva noi!>>\\
\indent Per sentire questo discorso migravano molte staffette, parecchi giorni prima; ed il giorno  dell'adunanza i contadini irreggimentati.\footnote{Si muovevano in molti per ascoltare ed il giorno dell'adunanza era tutti inquadrati e pronti all'ascolto} \\
\indent Si vede che mancavano persone a conteggiare quante suole delle scarpe... costava la concione.

 \begin{figure}[htb]
    \centering
    %\vspace{-0.7cm}
    \includegraphics[width=\textwidth]{meruzzi}
    \caption*{I farmacisti della farmacia comunale di via Mazzini. \\ Partendo da sinistra: Dr. \index[Personaggi]{Stella dott.}Stella, Dr. Cassiano \index[Personaggi]{Meruzzi dott. Cassiano}Meruzzi, Nando \index[Personaggi]{Isani Nando}Isani.\label{fig:meruzzi}}
    %\vspace{-0.3cm}
\end{figure}
