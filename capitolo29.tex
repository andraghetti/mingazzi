%-------------------------------------------------------------------------------
%	CAPITOLO 29
%-------------------------------------------------------------------------------

\chapter{I maestri di lingua - Araldo il Capolega B...}
\index[Personaggi]{Lanconelli Araldo}Araldo, seduto, calvo, con la coppole sulla cervice, ben nutrito, baffi, spioventi, alla moda, era un bell'uomo. Dritto non ci stava perché era zoppo... e così nelle concioni\footnote{Discorsi} dei repubblicani, si vestiva di nero e si metteva in evidenza nel palco accanto a \index[Personaggi]{Mirabelli Roberto (onorevole)}Mirabelli\footnote{Roberto Mirabelli, onorevole, di origine calabrese, più volte deputato di Ravenna per il PRI} o \index[Personaggi]{Mazzolani Ulderico (onorevole)}Mazzolani\footnote{Ulderico Mazzolani, onorevole repubblicano}, essendo un bel pezzo... decorativo.\\
Con le sue idee repubblicane non transigeva, era attaccato alla carica e fu anche assessore.\\
Una sera, durante l'assessorato, si recò a casa, come d'abitudine all'ora di cena, del \index[Personaggi]{Meruzzi Dr. Cassiano}Dr. Meruzzi\footnote{Dottor Cassiano Meruzzi, medico condotto di Alfonsine, cugino di Cassiano Bagnara, figlio di Giovanni, proprietario della casa di Vincenzo Monti}. Poveretto non stava più nella pelle, rideva, si contorceva, faceva delle smorfie.\\
Allora cominciò questo dialogo:\\
Dr. Meruzzi\index[Personaggi]{Meruzzi Dr. Cassiano}: "Insomma, sei troppo contento... hai qualche cosa, dillo."
\index[Personaggi]{Lanconelli Araldo}Araldo: "Ecco. Noi repubblicani, a fasèn un oratorio a què drì a la mura d'Terulin\footnote{"Noi repubblicani facciamo un oratorio qua dietro le mura di ??? CONTROLLARe"}".\\
Donne di casa: "Oh! Bene così andiamo a pregare... è vicino."\\
\index[Personaggi]{Lanconelli Araldo}Araldo si rabbuia e tace. 
\index[Personaggi]{Meruzzi Dr. Cassiano}Dr. Meruzzi: "Voi repubblicani, mangia preti, non ci sarebbe altro... che faceste proprio una chiesa."
\index[Personaggi]{Lanconelli Araldo}Araldo in fretta: "Ma che cisa, a fasèn un pisadur!\footnote{"Ma che chiesa, facciamo un orinatoio!"}".\\
\index[Personaggi]{Meruzzi Dr. Cassiano}Meruzzi e donne risposero con una risata e poi "Puh!"\\
\index[Personaggi]{Meruzzi Dr. Cassiano}Meruzzi: "Alora té da dì un orinatoio...\footnote{"Allora devi dire orinatoio"}"\\
\index[Personaggi]{Lanconelli Araldo}Araldo: "L'è l'istes, l'è question d'paròl"\footnote{"È lo stesso è questione di parole"}\\
Era anche assessore alla pubblica istruzione certamente se ne intendeva molto e poteva dire anche altro. Un altro giorno il nostro \index[Personaggi]{Lanconelli Araldo}Araldo, raccontava:\\
"L'ha scrèt a cà \index[Personaggi]{Forlivesi Sebastiano, "Nisò d'Furlivési" (commerciante)}Nisò d'Furlivési cl'à sintì \index[Personaggi]{De Maria Ugo (professore)}Ugo d'De Maria\footnote{Ugo De Maria, professore di lettere ad Alfonsine} in piaza a Palermo, che siringava la folla, contra a Nasi\footnote{"Ha scritto a casa Nisò d'Forlivesi che ha sentito Ugo De Maria in piazza a Palermo che siringava (per arringava) la folla"}" (era il tempo dello scandalo Nasi\footnote{Il Ministro della Pubblica Istruzione, Nunzio Nasi venne accusato di gravi irregolarità nel suo operato. Si parla di corruzione, sussidi ingiustificati, firme sospette, favoritismi, spese personali pagate con i soldi pubblici.})\\
Speriamo che il buon Dio tenga lontano ai palermitani il mal d'urina! Ce ne sarebbero troppe, ma finiamo con l'ultima.\\
Un giorno, al nostro Araldo viene consegnata la scheda per il censimento della popolazione. Da letterato si mette subito all'opera. \\
\\\\
\textcal \Huge	
Araldo L\:.\:.\:.\: fu \:.\:.\:.\:\\
\normalfont \normalsize
di professione:
\textcal \Huge Agente Urale
\normalfont \normalsize (per ‘rurale')
di religione: 
\textcal \Huge Ateo
\normalfont \normalsize\\\\
Speriamo bene che il nostro protagonista non vada alla storia... e che la scheda di suo pugno non vada in bella mostra... in vetrina.\\
\\
\centerline{\rule{1.5cm}{0.4pt}}\\
\\
Nel momento rosso i socialisti avevano caricato un buon uomo di molte frasi, capite come poteva e secondo lui solo\footnote{Interpretate a modo suo}.\\
Necessitava una concione\footnote{Discorso solenne in pubblico}. Il nostro uomo, rosso, forte, gesticolante cominciava.\\
"Soci, amici, compagni, sucilèsta, d'la sucietè, d'la fratelênza, d'la adunênza, d'la cumbuccila, d'la sozia di cuntadèn.\footnote{"Soci, amici, compagni, socialisti, della società, della fratellanza, dell'adunanza, della combucciella, della società dei contadini SISTEMARE"}" Punto e basta il repertorio era esaurito con un "Evviva noi!"\\
Per sentire questo discorso migravano molte staffette, parecchi giorni prima; ed il giorno  dell'adunanza i contadini irreggimentati.\footnote{Si muovevano in molti per ascoltare ed il giorno dell'adunanza era tutti inquadrati e pronti all'ascolto} \\
Si vede che mancavano persone a conteggiare quante suole delle scarpe... costava la concione.

