%-------------------------------------------------------------------------------
%	CAPITOLO 20
%-------------------------------------------------------------------------------

\chapter{Gli artigiani}

A dire il vero in Alfonsine non furono molti, si ricorda una epigrafe\footnote{Testo esposto pubblicamente su un supporto di materiale non deperibile} che andò per varie generazioni e cioè:
\\\\
\textcal \Huge
	\centerline{Gramantieri Tomaso}
	\centerline{e car fasè}
	\centerline{Santoni Proculo}
	\centerline{ul piturè}\normalfont \normalsize \footnote{Tomaso Gramantieri fece il carro, Proculo Santoni lo pitturò - Mi è stato detto che in una casa dietro alla \index[Luoghi]{Villa Marini}Villa Marini, vi era questa epigrafe che però riportava una frase leggermente diversa: "Checco Gramantieri e cas fasè, Brocul e Santoni il piturè"}

\index[Personaggi]{Gramantieri Tomaso}\index[Personaggi]{Santoni Proculo}