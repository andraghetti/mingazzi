%-------------------------------------------------------------------------------
%	PREFAZIONE 
%-------------------------------------------------------------------------------

\chapter*{Prefazione}
L'alone di mistero che avvolgeva la morte di Mingazzi mi appassionò a tal punto da spingermi a cercare di scoprire sempre più dettagli sulla storia della mia famiglia e del mio paese. Dopo la morte di mio nonno perlustrai ogni singolo centimetro del suo studio, in cui trovai dei manoscritti. Preso coscienza di ciò che rappresentavano, scelsi di condividerli con il mio paese.
\\Questi manoscritti trattano di storielle o meglio di personaggi particolari alfonsinesi. Per la maggior parte sono prese in giro di alfonsinesi realmente esistiti, mentre altre sono veri e propri scorci di vita alfonsinese dell'800 e del '900. A giudicare dal numero delle storielle, Mingazzi iniziò a scriverle pochi anni prima della fine della guerra. Nel leggerle, ci si accorgerà che i primi capitoli sono più dettagliati e risalenti al periodo ottocentesco, mentre le ultime sono più semplici, meno dettagliate e ambientate nel novecento. Questo, considerando la data di nascita di Mingazzi, fa pensare che le storielle gli siano state raccontate dal padre, o da persone più anziane di lui. Inoltre le ultime, benché siano presenti nell'indice, non sono mai state scritte per ovvi motivi. Ho deciso quindi di trascrivere tutte le storielle e di commentarle attingendo informazioni dal sito di Luciano Lucci \emph{alfonsinemonamour.racine.ra.it}, dal libro di Romano Pasi, \emph{Storia di Alfonsine} e dal libro di Giovanni e Maria Francesca Zanzi \emph{Le Alfonsine il volto e l'anima}. Benché siano storielle sciocche, presentano parecchi dettagli dell'Alfonsine prebellica e della vita di fine'800 e inizio '900 che sono, a mio avviso, \emph{inediti} se così si può dire.
