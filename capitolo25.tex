%-------------------------------------------------------------------------------
%	CAPITOLO 25
%-------------------------------------------------------------------------------

\chapter{I funerali di 1\textsuperscript{a}, 2\textsuperscript{a}, 3\textsuperscript{a}, 4\textsuperscript{a} classe}
Nella chiesa parrocchiale avevano sede le confraternite femminili e maschili che erano le più importanti. Vi era la compagnia del sacco, perché i suoi membri portavano una cappa con cappuccio nero e neanche la punta del naso lasciavano scorgere. La compagnia di S. Antonio, con saio bianco e mantellina verde.	\\
Altre confraternite tra le quali quella del SS. Sacramento i cui membri portavano la cappa bianca e sopra una mantellina rossa. Qualche vecchio esemplare si può ancora vedere di quest'ultima compagnia, tra i vecchi e nelle grandi funzioni religiose.\\
La compagnia di S. Antonio possedeva le bare per il trasporto a spalla dei defunti. I confratelli non pagavano nulla per i loro funerali: pensava la compagnia. La compagnia poi mandava ai funerali dei confratelli una sua rappresentanza numerosa vestita dei caratteristici costumi e con gonfalone\footnote{Stendardo della confraternita}.\\
Il funerale si disponeva con chierico e crocifisso in testa e clero. Poi venivano i confratelli in due fila, ai margini stradali, con torcia in mano e oranti. In mezzo della strada erano i gonfaloni delle confraternite. Poi la bara ed i portatori, sempre in numero di otto, 4 portatori e 4 ricambi. E dietro i parenti, amici, ecc. La bara nuova era la migliore con un grande panno nero arabescato, con teschio tibie ecc, ed era riservata ai funerali di 1ª classe. \\
La bara vecchia era più leggera, più stretta, portava un panno meno lussuoso ed era per i funerali di 2ª classe. Vi erano adibiti 4 portatori. Per i funerali di terza classe vi era il cataletto, specie di barella, con sopra un coperchio nero con sopra dipinta la morte e la croce. Serviva per i morti ed anche per portare i vivi malati all'ospedale.\\
I portatori erano due, senza ricambio, e per molti anni furono \index[Personaggi]{Loz}Loz, bracciante, pestatore del pepe nel mortaio dei vari negozi, e \index[Personaggi]{Paulon, e Sandron d'Schenal}Sandron d'Schenal, anelante\footnote{Aspirante} affossatore del becchino. Erano pagati 15 baiocchi ciascuno.\\
\index[Personaggi]{Loz}Loz e Paulon\footnote{Qui Mingazzi lo chiama Paulon, sopra Sandron. Sono sicuramente la stessa persona e probabilmente Paulon veniva anche chiamato "e sandron ad Schenal"}, staccavano la loro tracolla dalla carriola, la infilavano nelle stanghe del cataletto e via per la loro opera funebre.\\
Il clero era numeroso per i funerali di 1ª classe, lo sbattocchiamento delle campane grande, ceri ecc. Per quelli di seconda il clero era meno numeroso, meno i ceri ecc. Per la terza classe, un cappellano, in servizio gratuito ed il chierico in testa.\\
Al cimitero era un caso quasi impossibile che non succedesse una lite e non venissero alle mani i portatori, o parenti del morto con il prepotente becchino (\index[Personaggi]{Giuseppe, "Iusef" (becchino)}Iusef) il quale si credeva nelle sue funzioni un sovrano dispotico. Una volta aveva, questo malvagio, calato nella fossa un morto bocconi\footnote{In posizione distesa con la faccia in giù}. Alle rimostranze del cappellano (\index[Personaggi]{Rotondi (don)}Don Rotondi) rispose male, ma il cappellano, levò la croce dal manico... e con questo qui battè sul groppone di \index[Personaggi]{Giuseppe, "Iusef" (becchino)}Iusef.\\
Non era difficile vedere durante i funerali uno dei confratelli uscire dalle righe, discendere nel fosse per qualche occorrenza. una volta ci fu di peggio. A pochi metri dalla casa \index[Luoghi]{Dall'Ara (palazzo)}Dall'Ara sulla \index[Luoghi]{Raspona (via)} Raspona, uno dei primi confratelli si calò le brache... sul ciglio della strada ed in quella poetica situazione osservò la sfilata del funerale... come uno spettatore curioso.\\
I funerali di 4ª classe erano riservati agli annegati, colerosi, morti uccisi ecc. Erano caricati sulla rete di corda del biroccio del becchino coperti da una stuoia e via...\\
Sotto al papa i morti erano coperti dal un coppo sulla faccia, dopo vennero le casse... fino alle alterali lussuose.\footnote{Nel periodo in cui la Romagna era Stato Pontificio, si usava mettere solamente un coppo sulla faccia del morto; successivamente si utilizzarono le bare funebri. Infine si arrivò ad utilizzare anche le chiesette cimiteriali.}\\
Non solo si deve dire che la gente cambia mondo, ma la stessa gente... ha cambiato il mondo andando sfarzosamente al cimitero in automobile. 

