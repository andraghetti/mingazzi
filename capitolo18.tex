%-------------------------------------------------------------------------------
%	CAPITOLO 18
%-------------------------------------------------------------------------------

\chapter{Benedette le tue gambe}
Nel 1849 tra \index[Luoghi]{Alfonsine}Alfonsine e \index[Luoghi]{Fusignano}Fusignano si era mobilitata una compagnia di guardie nazionali, comandata da capitano avvocato \index[Personaggi]{Santoni Pietro (capitano, avvocato)} Santoni di Fusignano. Questo signore capitano era persona greve di peso, con una gran pancia e due gambe esili, come dicono.\\
\indent In rinforzo all'esercito regolare questa compagnia armata di schioppi da caccia, allora non avevano armi da guerra, pistole ecc. era stata spinta fino al Piave\footnote{Il Piave è un fiume italiano, che scorre interamente in Veneto nell'omonima valle} insieme all'esercito operante. Su questo fiume, sacro alla patria a sinistra erano gli austriaci, a destra gli Italiani.\\
\indent La brava compagnia della nostra guardia, parte era di guardia, parte a riposo sui cascinali\footnote{Gruppo di casolari in aperta campagna, organizzati o no come cascine.} vicini, quando un brutto mattino ebbe un brusco risveglio. L'artiglieria aveva smantellato due nostri pezzi di artiglieria, minacciava l'attacco delle fanterie. Stordite, le povere guardie si diedero alla fuga. I dormienti si svegliarono e poco rendendosi conto dell'accaduto seguirono i fuggiaschi.\\
\indent Tra le guardie militava anche un certo \index[Personaggi]{Taglioni}Taglioni di qui, famoso cacciatore addomesticatore di cani trifolai\footnote{Cani da tartufo}, secco, smilzo e celebre corridore. Il \index[Personaggi]{Taglioni}Taglioni dormiva su di un cascinale, al frastuono si svegliò, si trovò solo, solo, impaurito ancora di più perché sperduto, prese la fuga, raggiungendo or l'uno or l'altro dei commilitoni che sorpassava con una velocità elastica da saetta, senza curarsi di loro. Dopo poco raggiunse il capitano\index[Personaggi]{Santoni Pietro (capitano, avvocato)}, e quando lo ebbe sorpassato si sentì da questo chiamare, con la invocazione: <<E mi Taiòn, banadèt al tu gamb\footnote{<<Il mio Taglioni, benedette le tue gambe>>}>>.\\
\indent L'invocato, si soffermò e degnò di una frettolosa risposta curiosa il suo capitano: <<A se sgnor me m'avèg da què\footnote{<<Si signore, io me ne vado da qui>>}>>\\
\indent Accompagnò la detta con un colpo della mano sinistra sull'avanbraccio destro, molto significativa, e via nuovamente di corsa.\\
\indent Intanto il povero capitano\index[Personaggi]{Santoni Pietro (capitano, avvocato)}, ansante e sbuffante, col sudore caldo e freddo dell'emozione zampettava per mettere la maggior distanza tra di lui e l'odiato nemico.\\