%-------------------------------------------------------------------------------
%	CAPITOLO 28
%-------------------------------------------------------------------------------

\chapter{Uno stronzo nella pentola}
Il signor \index[Personaggi]{Camerani Matteo (farmacista)}Matteo C.\:.\:.\footnote{Matteo Camerani fu un farmacista e fu figlio di Giovan Antonio, governatore di Alfonsine e figlio di Maria Cristina Monti, sorella del poeta Vincenzo Monti. Può sembrare confusionaria la genealogia ma è di fatto: Matteo (fattore), Giovan Antonio (governatore), Matteo (farmacista), Giovan Antonio} aveva fatto scodellare la minestra per sé, la moglie, ed i 4 figli e con la moglie cominciarono a soffiare sulle prime cucchiaiate perché si raffreddasse.\\
I ragazzi erano a tavola meno uno, guardavano in giro e non mangiavano. Matteo, visto il posto vuoto: "Dov'è \index[Personaggi]{Camerani Giovan Antonio}Giov Antonio...?"\\
I ragazzi zitti e poi: "Un'iè\footnote{"Non c'è"}"\\
Matteo\index[Personaggi]{Camerani Matteo (farmacista)}, al servitore: "Galli val a ciamé, dov'è\footnote{"Vai a chiamarlo, dov'è"}"\\
Ragazzi: "L'è a là fura\footnote{"È là fuori!"}!".\\
Matteo\index[Personaggi]{Camerani Matteo (farmacista)}: "Perché non mangiate!".\\
I ragazzi zitti.\\
Matteo\index[Personaggi]{Camerani Matteo (farmacista)}:  "Magnì... av deg dal scòpul\footnote{"Mangiate... vi dò delle scopole" (scapaccioni)}".\\
I ragazzi tentano la fuga, ma sono fermati.\\
Matteo\index[Personaggi]{Camerani Matteo (farmacista)}: "Parchè an magnì\footnote{"Perché non mangiate?"}?"\\
I ragazzi timidamente: "... parchè Vàn Antoni\index[Personaggi]{Camerani Giovan Antonio}... la mès un strònz in t'la pignata\footnote{"Perché Giov Antonio ha messo uno stronzo nella pentola"}".\\
Matteo: "Ahc! Vigliac!\footnote{"Bleah! Vigliacco!"}" e buttò tutto all'aria.\\
\\
\centerline{\rule{1.5cm}{0.4pt}}\\
\\
\index[Personaggi]{Boari Attilio (farmacista)}Boari era farmacista col signor Matteo\index[Personaggi]{Camerani Matteo (farmacista)}. La mattina del sopracitato fatto, non si sentiva bene.\\
Per rinforzarlo sulle 11 gli portarono una tazza di buon brodo... di quello...\\
Torse la bocca e poi: "Cos'al ste brod. L'ha un fiè!\footnote{"Cos'ha questo brodo? Fa una certa puzza!"}"\\
Ma lo trangugiò egualmente... come ricostituente sostanzioso. I suoi clienti possano ritenersi vendicati... se da lui hanno avuto delle medicine amare!

