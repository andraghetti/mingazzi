%-------------------------------------------------------------------------------
%	CAPITOLO 2
%-------------------------------------------------------------------------------

\chapter{L'indoratore}
Nel \index[Luoghi]{Carraretto Fernè}Carraretto Fernè\footnote{Si intende il quartiere attorno al \textbf{Palazzo Fernè}, ancora oggi presente in via Mameli 3. \index[Personaggi]{Fernè Ferdinando}Ferdinando Fernè era un commerciante di vini all'ingrosso di Bologna ma che aveva un palazzo ad Alfonsine. Maggiori dettagli nell'immagine a fine storiella. (pag. \pageref{fig:villamassaroli})}, allora Massaroli\index[Personaggi]{Massaroli Paolo}, che va ai maceri, vi erano molte vecchie case prima del 1860. \index[Personaggi]{C. Francesco}Francesco C... figlio di buona famiglia nel fiore degli anni e con la blaga del liberalismo contro gli odiati papalini\footnote{Si intende la fissazione del liberalismo. Il liberalismo si ispira a ideali di tolleranza, libertà ed eguaglianza e contesta i privilegi dell'aristocrazia e del clero e l'origine divina del potere del sovrano.}, svelto ardito milite della guardia nazionale, andava a morosa da una tale del \index{Carraretto Fernè}Carraretto. \\
Il suo sentimentalismo aveva parecchie espressioni di adorazione contemplative per la sua bella.
Le scopriva... certe parti, poi battendovi sopra il palmo della mano con carezzevole posa e moine... non si poteva trattenere di esclamare:\\
\indent <<Poverina quanto sei bella, vorrei indorarti\footnote{Ricoprire d'oro}...>>\\
Dillo una sera, dillo due, la cosa andò agli orecchi di certi nottambuli sfaccendati e una sera che il nostro \index[Personaggi]{C. Francesco}Francesco C. era a morosa, sentì bruscamente bussare alla porta, mentre era nella solita contemplazione esilarante.
Si fece buio in faccia; erano i tempi dei ladri e delle bande di Altini\footnote{Altini era un noto brigante che assieme alla sua banda saccheggiava case e sterminava intere famiglie} e di altri e lui temeva... e brusco chiese: <<Chi è?>>\\
Una voce sonora dal di fuori rispose: <<L'indoratore\footnote{L'indoratore era un artigiano che ricopriva oggetti con un sottile strato d'oro}>> e subito a passo di corsa fuggì.
Il nostro buon \index[Personaggi]{C. Francesco}Francesco che aveva tremato alla bussata per la paura di un agguato di ladri, sentendo il nemico che fuggiva si rinfrancò e per la difesa del bel sesso inveì alle stelle: molti ‘boia' di qua, ‘porca' di la, con tutto il coraggio e la forza che può avere chi, deve fare bella figura senza più nulla temere.


 \begin{figure}[htb]
    \centering
    %\vspace{-0.7cm}
    \includegraphics[width=\textwidth]{villamassaroli}
    \caption[Palazzo Fernè - Villa Massaroli]{\textit{Il \textbf{Palazzo Fernè} detto anche \textbf{Villa Massaroli} ancora oggi (2017) presente in via Mameli 3. È attualmente disabitato, e di proprietà degli eredi Pirazzoli. Dalle mappe catastali napoleoniche (e poi gregoriane), risalenti alla prima metà dell'800, il palazzo risulta essere già costruito e di proprietà di \index[Personaggi]{Massaroli Paolo}Massaroli Paolo. I Ferné, proprietari di vaste tenute e di una villa a Lavezzola, si dedicavano soprattutto a produzione e commercio di vino. \index[Personaggi]{Fernè Ferdinando}\textbf{Ferdinando Fernè} sposò nel 1890 una figlia di Massaroli, di nome \index[Personaggi]{Massaroli Diana Anna Cristina `Aunita'}Diana Anna Cristina detta anche Aunita nata nel 1870. Per loro fu arredata la villa che era dei Massaroli e donata come dote alla nuova famiglia. Nel dicembre del 1890 nacque il primo ed unico figlio \index[Personaggi]{Fernè Vincenzo Enzo}Vincenzo Enzo Fernè, che diventerà poi il podestà di Bologna nel 1939. La villa era dotata di uno spazio adibito a cantina per le botti di vino e pare anche di una grande vasca sotterranea dove veniva depositato il vino prodotto e da smerciare a commercianti vari. }}
    \label{fig:villamassaroli}
    %\vspace{-0.3cm}
\end{figure}










































%